\documentclass[a4paper,11pt]{article}

\usepackage[top=1in,bottom=1in,]{geometry}
\usepackage{enumerate}
\usepackage{amsmath,amsfonts,amssymb}

\newcommand{\pr}{\mathrm{Pr}}
\newcommand{\nsp}{\mathbb{N}}
\newcommand{\rsp}{\mathbb{R}}
\newcommand{\zsp}{\mathbb{Z}}

\begin{document}

\title{\bf IERG 6130 (2015 Spring) - Problem Set 1}
\author{Instructor: Dahua Lin}
\date{}
\maketitle

\part{Markov Chain Theory}

\section*{Problem 1}

It is an interesting problem to study the chance of success in gambling. Suppose a player \textit{Bob} with fortune $k$ units plays a game with a rich adversary with fortune $m$, where both $k$ and $m$ are positive integers and $k < m$. At each step, Bob has probability $p$ of winninig one unit and probability $1 - p$ of losing one unit. The game stops until either Bob or his adversary loses all the fortune. Let $X_t$ be the units of fortune that Bob has after $t$-steps, then $(X_t)$ is a Markov chain with $X_0 = k$.

\begin{enumerate}[(a)]

\item Please specify the transition probability matrix $P$. 

\item Is this chain irreducible? 

\item A state $x$ is called \textit{absorbing} if $P(x, x) = 1$. Are there any absorbing states of this chain? If so, please specify them.

\item What's the probability that Bob loses the entire fortune at the end of the game? Please discuss this problem in three cases, $p = \frac{1}{2}$, $p < \frac{1}{2}$, and $p > \frac{1}{2}$. 

\item If the Bob's adversary is \textit{infinitely} rich, what's the probability that Bob loses the entire fortune? Again, there are three cases, $p = \frac{1}{2}$, $p < \frac{1}{2}$, and $p > \frac{1}{2}$.  

\end{enumerate}


\section*{Problem 2}

Consider a reflexive random walk over $\nsp$ with transition probability matrix $P$ as follows:
\[
P(x, x') = \begin{cases}
     p & (x' = x + 1) \\
     1 - p & (x > 0 \ \& \ x' = x - 1) \\
     1 - p & (x = 0 \ \& \ x' = x) \\
     0 & (\text{otherwise})
 \end{cases},
\]
where $0 < p < 1$. 

\begin{enumerate}[(a)]

\item Is this chain irreducible and aperiodic?

\item Please compute $P^{2n}(x, x)$. 

\item Please show that this chain is recurrent if and only if $p = \frac{1}{2}$.

\item Please show that this chain is positive recurrent when $p < \frac{1}{2}$ and null recurrent when $p = \frac{1}{2}$. 

\item Please compute the equilibrium distribution $\pi$ when $p < \frac{1}{2}$.

\end{enumerate}

\clearpage

\section*{Problem 3}

Consider the random walk process over $\zsp$ described as follows. Starting from $X_0 = 0$, the chain evolves as follows:
\[
\pr(X_{t+1} = x' | X_t = x) = \begin{cases}
     p & (x' = x + 1) \\
     1 - p & (x' = x - 1) \\
     0 & (\text{otherwise})
 \end{cases}.
\]

\begin{enumerate}[(a)]

\item Please show that this chain is recurrent if and only if $p = \frac{1}{2}$.

\item Please show that this chain is null-recurrent when $p = \frac{1}{2}$. 

\end{enumerate}


\section*{Problem 4}

Given a \textit{finite} Markov chain with transition probability matrix $P$, please show the following:

\begin{enumerate}[(a)]

\item If the chain is irreducible and aperiodic, please show that $P^n$ is positive, \textit{i.e.}~all elements are positive, for some $n \in \nsp$. 

\item If the chain is aperiodic, please show that $P^n$ is not a positive matrix, \textit{i.e.}~at least one element is zero, for every $n \in \nsp$.

\item If the chain is reducible, please show that $P^n$ is not a positive matrix, for every $n \in \nsp$. 

\end{enumerate}


\section*{Problem 5}

Consider a Markov chain on $\mathbb{R}$ with $X_{t+1} | X_t \sim \mathrm{Uniform}(X_t - 1, X_t + 1)$. Please show that this chain is $\varphi$-irreducible \textit{w.r.t.}~the Lebesgue measure $\varphi$.

\noindent \textbf{Note:} To show this, it suffices to show that starting from any $x \in \rsp$, the probability of entering any given interval $(a, b)$ in finite steps is positive. 


\part{Basics of Monte Carlo Methods}







\end{document}


